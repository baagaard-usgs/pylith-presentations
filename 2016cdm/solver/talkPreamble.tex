\mode<presentation>
{
  %%\usetheme[right]{Hannover}
  %%\usetheme{Goettingen}
  %%\usetheme{Dresden}
  \usetheme{Warsaw}
  \useoutertheme{infolines}
  \useinnertheme{rounded}

  \setbeamercovered{transparent}
}

\usepackage[english]{babel}
\usepackage[latin1]{inputenc}
\usepackage{times}
\usepackage[T1]{fontenc} % Note that the encoding and the font should match. If T1 does not look nice, try deleting this line
\usepackage{amsfonts, amsmath, subfigure, multirow}
\usepackage{hyperref}
\newlength\LL \settowidth\LL{1000}

%\usepackage{beamerseminar}
\usepackage{graphicx}
\usepackage{array}
\usepackage{ulem}
\usepackage{listings}
\usepackage{xfrac}
\usepackage{xspace}
\usepackage{colortbl}
\usepackage{alltt}

\usepackage{mathtools}       % For \mathllap
\usepackage[boxed]{algorithm}
\usepackage{algpseudocode}   % For \Procedure on algorithm

\usepackage{tikz}
\usepackage{pgflibraryshapes}
\usetikzlibrary{decorations.text}
\usetikzlibrary{decorations.pathreplacing}
\usetikzlibrary{backgrounds}
\usetikzlibrary{calc}
\usetikzlibrary{arrows}
\usetikzlibrary{snakes}
\usetikzlibrary{positioning}
\definecolor{cffffff}{RGB}{255,255,255}
\definecolor{cff00c0}{RGB}{255,0,192}
\definecolor{c0f00ff}{RGB}{15,0,255}
\definecolor{c00ff20}{RGB}{0,255,32}
\definecolor{cfffd00}{RGB}{255,253,0}
\definecolor{c00e0ff}{RGB}{0,224,255}

\usepackage{pgfplots}
\usepackage{pgfplotstable}

\usepackage[absolute,overlay]{textpos}
\setlength{\TPHorizModule}{\textwidth}
\setlength{\TPVertModule}{\textheight}
\textblockorigin{0pt}{0pt} % start everything near the top-left corner

% Text macros
\makeatletter
\DeclareRobustCommand\onedot{\futurelet\@let@token\@onedot}
\def\@onedot{\ifx\@let@token.\else.\null\fi\xspace}
\def\eg{{e.g}\onedot} \def\Eg{{E.g}\onedot}
\def\ie{{i.e}\onedot} \def\Ie{{I.e}\onedot}
\def\cf{{c.f}\onedot} \def\Cf{{C.f}\onedot}
\def\etc{{etc}\onedot}
\def\vs{{vs}\onedot}
\def\wrt{w.r.t\onedot}
\def\dof{d.o.f\onedot}
\def\etal{{et al}\onedot}
\makeatother

\newenvironment{changemargin}[2]{%
  \begin{list}{}{%
    \setlength{\topsep}{0pt}%
    \setlength{\leftmargin}{#1}%
    \setlength{\rightmargin}{#2}%
    \setlength{\listparindent}{\parindent}%
    \setlength{\itemindent}{\parindent}%
    \setlength{\parsep}{\parskip}%
  }%
  \item[]}{\end{list}}
\newcommand{\D}{\mathcal{D}}
\newcommand{\E}{\mathcal{E}}
\newcommand{\F}{\mathcal{F}}
\newcommand{\I}{\mathcal{I}}
\newcommand{\M}{\mathcal{M}}
\newcommand{\N}{\mathcal{N}}
\newcommand{\bigO}{\mathcal{O}}
\newcommand{\Q}{\mathcal{Q}}
\newcommand{\R}{\mathbb{R}}
\newcommand{\X}{\mathbb{X}}
\newcommand{\kb}{\tt}
\newcommand{\vF}{\Vec{F}}
\newcommand{\vJ}{\Vec{J}}
\newcommand{\vM}{\Vec{M}}
\newcommand{\vN}{\Vec{N}}
\newcommand{\vb}{\Vec{b}}
\newcommand{\vd}{\Vec{d}}
\newcommand{\vf}{\Vec{f}}
\newcommand{\vj}{\Vec{j}}
\newcommand{\vk}{\Vec{k}}
\newcommand{\vn}{\Vec{n}}
\newcommand{\vr}{\Vec{r}}
\newcommand{\vu}{\Vec{u}}
\newcommand{\vv}{\Vec{v}}
\newcommand{\vx}{\Vec{x}}
\newcommand{\vy}{\Vec{y}}
\newcommand{\vz}{\Vec{z}}
\newcommand\oneitem[1]{\begin{itemize} \item #1 \end{itemize}}
\def\shell#1{{\tt \$ #1}}

%   Default fixed font does not support bold face
\DeclareFixedFont{\ttb}{T1}{txtt}{bx}{n}{12} % for bold
\DeclareFixedFont{\ttm}{T1}{txtt}{m}{n}{12}  % for normal

% Color macros
\newcommand{\black}{\textcolor{black}}
\newcommand{\blue}{\textcolor{blue}}
\newcommand{\green}{\textcolor{green}}
\newcommand{\red}{\textcolor{red}}
\newcommand{\brown}{\textcolor{brown}}
\newcommand{\cyan}{\textcolor{cyan}}
\newcommand{\magenta}{\textcolor{magenta}}
\newcommand{\yellow}{\textcolor{yellow}}

\usepackage{color}
\definecolor{deepblue}{rgb}{0,0,0.5}
\definecolor{deepred}{rgb}{0.6,0,0}
\definecolor{deepgreen}{rgb}{0,0.5,0}

% Nonlinear preconditioning
\newcommand{\vecname}[1]{\ensuremath{\mathbf{#1}}}
\newcommand{\restrict}{\vecname{R}}
\newcommand{\interp}{\vecname{P}}
\newcommand{\inject}{\vecname{\widehat{R}}}
\newcommand{\pc}{\ensuremath{-}}
\newcommand{\lp}{\ensuremath{\pc_{L}}}
\newcommand{\rp}{\ensuremath{\pc_{R}}}
\newcommand{\lin}{\ensuremath{\backslash}}
\newcommand{\solvername}[1]{\ensuremath{\begingroup\text{#1}\endgroup}}
\newcommand {\NRICH}{\solvername{NRICH}}
\newcommand {\NGMRES}{\solvername{NGMRES}}
\newcommand {\NEWT}{\mathcal{N}}
\newcommand {\NCG}{\solvername{NCG}}
\newcommand {\NGS}{\solvername{NGS}}
\newcommand {\FAS}{\solvername{FAS}}
\newcommand {\QN}{\solvername{QN}}
\newcommand {\NASM}{\solvername{NASM}}
\newcommand {\RAS}{\solvername{RAS}}
\newcommand {\ASM}{\solvername{ASM}}
\newcommand {\MG}{\solvername{MG}}
\newcommand {\GSN}{\solvername{GSN}}
\newcommand {\GS}{\solvername{GS}}
\newcommand {\GSPIN}{\solvername{GSPIN}}
\newcommand {\SOR}{\solvername{SOR}}
\newcommand {\ASPIN}{\solvername{ASPIN}}
\newcommand {\CG}{\solvername{CG}}
\newcommand {\GMRES}{\solvername{GMRES}}
\newcommand {\krylov}{\solvername{K}}
\newcommand {\FGMRES}{\solvername{FGMRES}}
\newcommand {\ILU}{\solvername{ILU}}
\newcommand {\LU}{\solvername{LU}}
\newcommand {\NK}{\NEWT\lin\krylov}
\newcommand {\AND}{\solvername{Anderson}}
\newcommand {\GB}{\solvername{GB}}

% Optimization macros
\newcommand{\mini}{\mathop{\rm minimize}}
\newcommand{\st}{\mbox{subject to }}

% PDE macros
\newcommand{\lap}[1]{\Delta #1}
\newcommand{\grad}[1]{\nabla #1}
\renewcommand{\div}[1]{\nabla \cdot #1}

% Tutorial code macros
\def\code#1{{\tt #1}}
\def\enum#1{{\sc #1}}
\def\class#1{{\tt\bf #1}}
\def\function#1{\code{#1}}
\def\Update#1{\frametitle{Code Update} \begin{center}\Huge Update to {Revision \green{#1}}\end{center}}

% Python style for highlighting
\newcommand\pythonstyle{\lstset{
language=Python,
basicstyle=\ttm\scriptsize,
otherkeywords={self},             % Add keywords here
keywordstyle=\ttb\scriptsize\color{deepblue},
emph={MyClass,__init__},          % Custom highlighting
emphstyle=\ttb\scriptsize\color{deepred},    % Custom highlighting style
commentstyle=\scriptsize\color{deepgreen},
frame=tb,                         % Any extra options here
showstringspaces=false            % 
}}


% Python environment
\lstnewenvironment{python}[1][]
{
\pythonstyle
\lstset{#1}
}
{}

% Python for external files
\newcommand\pythonexternal[2][]{{
\pythonstyle
\lstinputlisting[#1]{#2}}}

% Python for inline
\newcommand\pythoninline[1]{{\pythonstyle\lstinline!#1!}}

% C style for highlighting
\newcommand\cstyle{\lstset{
language=C,
basicstyle=\ttm\scriptsize,
otherkeywords={MPI_Comm,TS,SNES,KSP,NPC,PC,DM,Mat,Vec,VecScatter,IS,PetscSF,PetscSection,PetscObject,PetscInt,PetscScalar,PetscReal,PetscBool,InsertMode,PetscErrorCode}, % Add keywords here
keywordstyle=\ttb\scriptsize\color{deepblue},
emph={PETSC_COMM_WORLD,PETSC_NULL,SNES_NGMRES_RESTART_PERIODIC},          % Custom highlighting
emphstyle=\ttb\scriptsize\color{deepred},    % Custom highlighting style
commentstyle=\scriptsize\color{brown},
stringstyle=\ttm\scriptsize\color{deepgreen},
frame=tb,                         % Any extra options here
showstringspaces=false            % 
}}

% C environment
\lstnewenvironment{cprog}[1][]
{
\cstyle
\lstset{#1}
}
{}

% C for external files
\newcommand\cexternal[2][]{{
\cstyle
\lstinputlisting[#1]{#2}}}

% C for inline
\newcommand\cinline[1]{{\cstyle\lstinline!#1!}}

% bash style for highlighting
\newcommand\bashstyle{\lstset{
language=bash,
basicstyle=\scriptsize\ttfamily,
}}

% bash environment
\lstnewenvironment{bash}[1][]
{
\bashstyle
\lstset{#1}
}
{}

% bash for inline
\newcommand\bashinline[1]{{\bashstyle\lstinline!#1!}}

% C++ style for highlighting
\newcommand\cppstyle{\lstset{
language=C++,
basicstyle=\ttm\scriptsize,
otherkeywords={MPI_Comm,TS,SNES,KSP,PC,DM,Mat,Vec,VecScatter,IS,PetscSF,PetscSection,PetscObject,PetscInt,PetscScalar,PetscReal,PetscBool,InsertMode,PetscErrorCode}, % Add keywords here
keywordstyle=\ttb\scriptsize\color{deepblue},
emph={PETSC_COMM_WORLD,PETSC_NULL,SNES_NGMRES_RESTART_PERIODIC},          % Custom highlighting
emphstyle=\ttb\scriptsize\color{deepred},    % Custom highlighting style
commentstyle=\scriptsize\color{brown},
stringstyle=\ttm\scriptsize\color{deepgreen},
frame=tb,                         % Any extra options here
showstringspaces=false            % 
}}

% C++ environment
\lstnewenvironment{cpp}[1][]
{
\cppstyle
\lstset{#1}
}
{}

% Fortran style for highlighting
\newcommand\fortranstyle{\lstset{
language=Fortran,
basicstyle=\ttm\scriptsize,
otherkeywords={MPI_Comm,TS,SNES,KSP,PC,DM,Mat,Vec,VecScatter,IS,PetscSF,PetscSection,PetscObject,PetscInt,PetscScalar,PetscReal,PetscBool,InsertMode,PetscErrorCode}, % Add keywords here
keywordstyle=\ttb\scriptsize\color{deepblue},
emph={PETSC_COMM_WORLD,PETSC_NULL,SNES_NGMRES_RESTART_PERIODIC},          % Custom highlighting
emphstyle=\ttb\scriptsize\color{deepred},    % Custom highlighting style
commentstyle=\scriptsize\color{brown},
stringstyle=\ttm\scriptsize\color{deepgreen},
frame=tb,                         % Any extra options here
showstringspaces=false            % 
}}

% Fortran environment
\lstnewenvironment{fortran}[1][]
{
\fortranstyle
\lstset{#1}
}
{}

% Make style for highlighting
\newcommand\makestyle{\lstset{
language=Make,
basicstyle=\ttm\scriptsize,
otherkeywords={MPI_Comm,TS,SNES,KSP,PC,DM,Mat,Vec,VecScatter,IS,PetscSF,PetscSection,PetscObject,PetscInt,PetscScalar,PetscReal,PetscBool,InsertMode,PetscErrorCode}, % Add keywords here
keywordstyle=\ttb\scriptsize\color{deepblue},
emph={PETSC_COMM_WORLD,PETSC_NULL,SNES_NGMRES_RESTART_PERIODIC},          % Custom highlighting
emphstyle=\ttb\scriptsize\color{deepred},    % Custom highlighting style
commentstyle=\scriptsize\color{brown},
stringstyle=\ttm\scriptsize\color{deepgreen},
frame=tb,                         % Any extra options here
showstringspaces=false            % 
}}

% Make environment
\lstnewenvironment{make}[1][]
{
\makestyle
\lstset{#1}
}
{}

\AtBeginSection[]
{
  \begin{frame}<beamer>
    \frametitle{Outline}
    \tableofcontents[currentsection,hideothersubsections]
  \end{frame}
}

\AtBeginSubsection[]
{
  \begin{frame}<beamer>
    \frametitle{Outline}
    \tableofcontents[sectionstyle=show/hide,subsectionstyle=show/shaded/hide]
  \end{frame}
}
